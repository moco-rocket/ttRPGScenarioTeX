% LuaLaTeX文書; UTF-8
\documentclass[twocolumn, lualatex, ja=standard]{bxjsarticle}
% \documentclass[a5paper,twocolumn,twoside, lualatex, ja=standard]{bxjsarticle} //A5版
\usepackage{luatexja-fontspec}
\usepackage{ftnright}
\usepackage{setting}

% font
\usepackage[haranoaji,nfssonly]{luatexja-preset}
\setmainjfont{HaranoAjiMincho-Regular.otf}[AutoFakeSlant=0.25]
\setsansjfont{HaranoAjiGothic-Regular.otf}[AutoFakeSlant=0.25]

% ダイスロール
\newcommand{\dice}[1][目星]{〈#1〉}

% ハイパーリンク
\usepackage[unicode=true]{hyperref}

\title{新クトゥルフ神話TRPGシナリオ\\
ここにタイトルを書きます} 
\author{moco} 
\date{} 

\begin{document}

\maketitle %タイトルを挿入
\textit{``ニャルラトホテプ……這い寄る混沌……残ったのはもうわたしだけ……この何もない空を聞き手にして、お話ししようと思います。''\\――H.P.ラヴクラフト}

\tableofcontents %目次を挿入

\section{シナリオ概要}
\subsection{はじめに}
このシナリオは,``新クトゥルフ神話TRPGルールブック''および``新クトゥルフ神話TRPG クトゥルフ2020''の選択ルール「年少探索者の創造とプレイ」に対応したシナリオで,年少探索者3〜4人向けにデザインされている.

探索者の年齢は小学生である7〜12歳の範囲で決めることになる.

プレイ時間は探索者の作成を含まずに3時間程度だろう.

\subsection{キーパー向け情報}
昨日、近所の吉野家行ったんです。吉野家。\footnote{ここに脚注を入れる}

そしたらなんか人がめちゃくちゃいっぱいで座れないんです\footnote{ここに脚注を入れる}
で、よく見たらなんか垂れ幕下がってて、150円引き、とか書いてあるんです。\footnote{ここに脚注を入れる}

もうね、アホかと。馬鹿かと。\footnote{ここに脚注を入れる}
お前らな、150円引き如きで普段来てない吉野家に来てんじゃねーよ、ボケが。
150円だよ、150円。
なんか親子連れとかもいるし。
一家4人で吉野家か。おめでてーな。
よーしパパ特盛頼んじゃうぞー、とか言ってるの。もう見てらんない。
お前らな、150円やるからその席空けろと。はい.

\subsection{主なNPC}

\verb|\|dice[ほにゃらら]と書くことによって自動でヤマカッコ(〈〉)で
括れるようになっています.こんなふうに.

「彼を安心させて話を聞き出すには,\dice[動物使い]あるいは\dice[爆破]の技能に成功する必要がある.」

Texなのでもちろん外部リンクを貼ったり,表を書いたり,\textbf{太字}

☃っ☃☃☃っ☃っ☃~~♪

\section{はじまり:深夜の吉野家}
% \addcontentsline{toc}{chapter}{はじめに}

昨日、近所の吉野家行ったんです。吉野家。

そしたらなんか人がめちゃくちゃいっぱいで座れないんです
で、よく見たらなんか垂れ幕下がってて、150円引き、とか書いてあるんです。

もうね、アホかと。馬鹿かと。
お前らな、150円引き如きで普段来てない吉野家に来てんじゃねーよ、ボケが。
150円だよ、150円。
なんか親子連れとかもいるし。
一家4人で吉野家か。おめでてーな。
よーしパパ特盛頼んじゃうぞー、とか言ってるの。もう見てらんない。
お前らな、150円やるからその席空けろと。はい.

昨日、近所の吉野家行ったんです。吉野家。

そしたらなんか人がめちゃくちゃいっぱいで座れないんです
で、よく見たらなんか垂れ幕下がってて、150円引き、とか書いてあるんです。

もうね、アホかと。馬鹿かと。
お前らな、150円引き如きで普段来てない吉野家に来てんじゃねーよ、ボケが。
150円だよ、150円。
なんか親子連れとかもいるし。
一家4人で吉野家か。おめでてーな。
よーしパパ特盛頼んじゃうぞー、とか言ってるの。もう見てらんない。
お前らな、150円やるからその席空けろと。はい.
昨日、近所の吉野家行ったんです。吉野家。

そしたらなんか人がめちゃくちゃいっぱいで座れないんです
で、よく見たらなんか垂れ幕下がってて、150円引き、とか書いてあるんです。

もうね、アホかと。馬鹿かと。
お前らな、150円引き如きで普段来てない吉野家に来てんじゃねーよ、ボケが。
150円だよ、150円。
なんか親子連れとかもいるし。
一家4人で吉野家か。おめでてーな。
よーしパパ特盛頼んじゃうぞー、とか言ってるの。もう見てらんない。
お前らな、150円やるからその席空けろと。はい.
昨日、近所の吉野家行ったんです。吉野家。

そしたらなんか人がめちゃくちゃいっぱいで座れないんです
で、よく見たらなんか垂れ幕下がってて、150円引き、とか書いてあるんです。

もうね、アホかと。馬鹿かと。
お前らな、150円引き如きで普段来てない吉野家に来てんじゃねーよ、ボケが。
150円だよ、150円。
なんか親子連れとかもいるし。
一家4人で吉野家か。おめでてーな。
よーしパパ特盛頼んじゃうぞー、とか言ってるの。もう見てらんない。
お前らな、150円やるからその席空けろと。はい.
昨日、近所の吉野家行ったんです。吉野家。

そしたらなんか人がめちゃくちゃいっぱいで座れないんです
で、よく見たらなんか垂れ幕下がってて、150円引き、とか書いてあるんです。

もうね、アホかと。馬鹿かと。
お前らな、150円引き如きで普段来てない吉野家に来てんじゃねーよ、ボケが。
150円だよ、150円。
なんか親子連れとかもいるし。
一家4人で吉野家か。おめでてーな。
よーしパパ特盛頼んじゃうぞー、とか言ってるの。もう見てらんない。
お前らな、150円やるからその席空けろと。はい.
昨日、近所の吉野家行ったんです。吉野家。

そしたらなんか人がめちゃくちゃいっぱいで座れないんです
で、よく見たらなんか垂れ幕下がってて、150円引き、とか書いてあるんです。

もうね、アホかと。馬鹿かと。
お前らな、150円引き如きで普段来てない吉野家に来てんじゃねーよ、ボケが。
150円だよ、150円。
なんか親子連れとかもいるし。
一家4人で吉野家か。おめでてーな。
よーしパパ特盛頼んじゃうぞー、とか言ってるの。もう見てらんない。
お前らな、150円やるからその席空けろと。はい.
☃っ☃☃☃っ☃っ☃~~♪

\textit{Italic(日本語非対応)}

``引用符(左側には「`」を使い,右側には「'」を使う)''

\end{document}