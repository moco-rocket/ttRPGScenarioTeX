% LuaLaTeX文書; UTF-8
\documentclass[twocolumn, lualatex, ja=standard]{bxjsarticle}
\usepackage{luatexja-fontspec}
\usepackage{ftnright}
\usepackage{setting}

% font
\usepackage[haranoaji,nfssonly]{luatexja-preset}
\setmainjfont{HaranoAjiMincho-Regular.otf}[AutoFakeSlant=0.25]
\setsansjfont{HaranoAjiGothic-Regular.otf}[AutoFakeSlant=0.25]

% 表
\usepackage{booktabs}

% ダイスロール
\newcommand{\dice}[1][目星]{〈#1〉}

% ハイパーリンク
\usepackage[unicode=true]{hyperref}

\title{新クトゥルフ神話TRPGシナリオ\\
ここにタイトルを書きます}
\author{moco}
\date{}

\begin{document}

\maketitle %タイトルを挿入
\textit{``ニャルラトホテプ……這い寄る混沌……残ったのはもうわたしだけ……この何もない空を聞き手にして、お話ししようと思います。''\\――H.P.ラヴクラフト,\ NYARLATHOTEP, 大久保ゆう訳}

\tableofcontents %目次を挿入

\section{シナリオ概要}
\subsection{はじめに}
このシナリオは,``新クトゥルフ神話TRPGルールブック''および``新クトゥルフ神話TRPG クトゥルフ2020''の選択ルール「年少探索者の創造とプレイ」に対応したシナリオで,年少探索者3〜4人向けにデザインされている.

探索者の年齢は小学生である7〜12歳の範囲で決めることになる.

プレイ時間は探索者の作成を含まずに3時間程度だろう.

\subsection{キーパー向け情報}
あああああああああああああああああああああああああああああああああああああああああああああああああああああああああああああああああああああああああああああああああああああああああああああああああああ

\subsection{主な機能}

\verb|\|dice[ほにゃらら]と書くことによって自動でヤマカッコ(〈〉)で
括れるようになっています.こんなふうに.\\

「彼を安心させて話を聞き出すには,\dice[動物使い]あるいは\dice[爆破]の技能に成功する必要がある.」\\

\TeX なのでもちろん\href{https://google.com}{外部リンク}を貼ったり,
\hypertarget{link1}{記事内にリンクポイント名を付与して}
\hyperlink{link1}{別の場所から参照}できるようにしたり,
表を書いたり(表\ref{table:table_1}),

\begin{table}
    \centering
    \caption{Microdoft Office Wordと\TeX の比較 \label{table:table_1}}
        \begin{tabular}{ccccc}
        \toprule
                    & \textgt{動作} &\textgt{値段} & \textgt{インターフェース} & \textgt{導入}  \\
        \midrule
        Word        & 重い          & 有料         & GUI                     & 簡単            \\
        \TeX        & 軽い          & 無料         & CUI                     & 面倒            \\
        \bottomrule
        \end{tabular}
\end{table}

脚注を入れたり\footnote{脚注とは本文の下の方につける注記のことである.},
文字を\textgt{太字}にしたり、あるいは\textit{斜体}にしたりもできます.\\

何よりも破壊的編集が容易なのがいいよね~!

\section{はじまり:深夜の吉野家}
% \addcontentsline{toc}{chapter}{はじめに}

昨日、近所の吉野家行ったんです。吉野家。

そしたらなんか人がめちゃくちゃいっぱいで座れないんです
で、よく見たらなんか垂れ幕下がってて、150円引き、とか書いてあるんです。

もうね、アホかと。馬鹿かと。
お前らな、150円引き如きで普段来てない吉野家に来てんじゃねーよ、ボケが。
150円だよ、150円。
なんか親子連れとかもいるし。
一家4人で吉野家か。おめでてーな。
よーしパパ特盛頼んじゃうぞー、とか言ってるの。もう見てらんない。
お前らな、150円やるからその席空けろと。はい.

昨日、近所の吉野家行ったんです。吉野家。

そしたらなんか人がめちゃくちゃいっぱいで座れないんです
で、よく見たらなんか垂れ幕下がってて、150円引き、とか書いてあるんです。

もうね、アホかと。馬鹿かと。
お前らな、150円引き如きで普段来てない吉野家に来てんじゃねーよ、ボケが。
150円だよ、150円。
なんか親子連れとかもいるし。
一家4人で吉野家か。おめでてーな。
よーしパパ特盛頼んじゃうぞー、とか言ってるの。もう見てらんない。
お前らな、150円やるからその席空けろと。はい.
昨日、近所の吉野家行ったんです。吉野家。

そしたらなんか人がめちゃくちゃいっぱいで座れないんです
で、よく見たらなんか垂れ幕下がってて、150円引き、とか書いてあるんです。

もうね、アホかと。馬鹿かと。
お前らな、150円引き如きで普段来てない吉野家に来てんじゃねーよ、ボケが。
150円だよ、150円。
なんか親子連れとかもいるし。
一家4人で吉野家か。おめでてーな。
よーしパパ特盛頼んじゃうぞー、とか言ってるの。もう見てらんない。
お前らな、150円やるからその席空けろと。はい.
昨日、近所の吉野家行ったんです。吉野家。

そしたらなんか人がめちゃくちゃいっぱいで座れないんです
で、よく見たらなんか垂れ幕下がってて、150円引き、とか書いてあるんです。

もうね、アホかと。馬鹿かと。
お前らな、150円引き如きで普段来てない吉野家に来てんじゃねーよ、ボケが。
150円だよ、150円。
なんか親子連れとかもいるし。
一家4人で吉野家か。おめでてーな。
よーしパパ特盛頼んじゃうぞー、とか言ってるの。もう見てらんない。
お前らな、150円やるからその席空けろと。はい.
昨日、近所の吉野家行ったんです。吉野家。

そしたらなんか人がめちゃくちゃいっぱいで座れないんです
で、よく見たらなんか垂れ幕下がってて、150円引き、とか書いてあるんです。

もうね、アホかと。馬鹿かと。
お前らな、150円引き如きで普段来てない吉野家に来てんじゃねーよ、ボケが。
150円だよ、150円。
なんか親子連れとかもいるし。
一家4人で吉野家か。おめでてーな。
よーしパパ特盛頼んじゃうぞー、とか言ってるの。もう見てらんない。
お前らな、150円やるからその席空けろと。はい.
昨日、近所の吉野家行ったんです。吉野家。

そしたらなんか人がめちゃくちゃいっぱいで座れないんです
で、よく見たらなんか垂れ幕下がってて、150円引き、とか書いてあるんです。

もうね、アホかと。馬鹿かと。
お前らな、150円引き如きで普段来てない吉野家に来てんじゃねーよ、ボケが。
150円だよ、150円。
なんか親子連れとかもいるし。
一家4人で吉野家か。おめでてーな。
よーしパパ特盛頼んじゃうぞー、とか言ってるの。もう見てらんない。
お前らな、150円やるからその席空けろと。はい.
☃っ☃☃☃っ☃っ☃~~♪

☃っ☃☃☃っ☃っ☃~~♪

\textit{Italic(日本語非対応)}

``引用符(左側には「`」を使い,右側には「'」を使う)''

\end{document}